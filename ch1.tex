% !TEX TS-program = pdflatex
% !TEX encoding = UTF-8 Unicode

% This is a simple template for a LaTeX document using the "article" class.
% See "book", "report", "letter" for other types of document.

\documentclass[11pt]{article} % use larger type; default would be 10pt

\usepackage[utf8]{inputenc} % set input encoding (not needed with XeLaTeX)

%%% Examples of Article customizations
% These packages are optional, depending whether you want the features they provide.
% See the LaTeX Companion or other references for full information.

%%% PAGE DIMENSIONS
\usepackage{geometry} % to change the page dimensions
\geometry{a4paper} % or letterpaper (US) or a5paper or....
% \geometry{margin=2in} % for example, change the margins to 2 inches all round
% \geometry{landscape} % set up the page for landscape
%   read geometry.pdf for detailed page layout information

\usepackage{graphicx} % support the \includegraphics command and options

% \usepackage[parfill]{parskip} % Activate to begin paragraphs with an empty line rather than an indent

%%% PACKAGES
\usepackage{booktabs} % for much better looking tables
\usepackage{array} % for better arrays (eg matrices) in maths
\usepackage{paralist} % very flexible & customisable lists (eg. enumerate/itemize, etc.)
\usepackage{verbatim} % adds environment for commenting out blocks of text & for better verbatim
\usepackage{subfig} % make it possible to include more than one captioned figure/table in a single float
% These packages are all incorporated in the memoir class to one degree or another...

%%% HEADERS & FOOTERS
\usepackage{fancyhdr} % This should be set AFTER setting up the page geometry
\pagestyle{fancy} % options: empty , plain , fancy
\renewcommand{\headrulewidth}{0pt} % customise the layout...
\lhead{}\chead{}\rhead{}
\lfoot{}\cfoot{\thepage}\rfoot{}

%%% SECTION TITLE APPEARANCE
\usepackage{sectsty}
\allsectionsfont{\sffamily\mdseries\upshape} % (See the fntguide.pdf for font help)
% (This matches ConTeXt defaults)

%%% END Article customizations

\usepackage{amsmath}

%%% The "real" document content comes below...


\begin{document}

\begin{center}
\LARGE Reddit LADR Study Group \\
Chapter 1
\end{center}

{\Large 1.A}

{\large Problem 1. }

\begin{align*}
	\frac{1}{a+bi}\cdot \frac{a-bi}{a-bi} = \frac{a-bi}{a^2+b^2} = \frac{a}{a^2+b^2}-\frac{b}{a^2+b^2}i
\end{align*}

so we take $c= \frac{a}{a^2+b^2}$ and similar for $d$.  Note that $a^2+b^2 = |a+bi|^2$ and it is often helpful to think of complex operations in terms of the modulus.  In effect what we've seen is $1/z = \frac{\overline{z}}{z\overline{z}} = \frac{\overline{z}}{|z|^2}$.

\pagebreak

{\Large 1.B}

{\large Problem 1.}

From the definitions this requires showing that $\vec v$ is the element such that if added to $-\vec v$ the result is $\vec 0$.  But this just follows again from the definition.

\pagebreak

{\Large 1.C}

{\large Problem 1.}

(a) and (d) are subspaces, the others are not.  You can rule out (b) because it does not contain $\vec 0$.  You can rule out (c) by additivity: $\begin{bmatrix} 1\\1\\0\end{bmatrix}, \begin{bmatrix}0\\1\\1\end{bmatrix}$ are each in the set but their sum is not.
\end{document}
